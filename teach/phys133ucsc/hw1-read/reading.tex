\documentclass[12pt]{article}

\usepackage[utf8]{inputenc}
\usepackage[top=.8in, bottom=.7in, left=.8in, right=.8in]{geometry}

\usepackage{amsmath,amssymb,amsthm}
\usepackage{graphicx}
\usepackage{xcolor}

\usepackage{hyperref}
\hypersetup{
    colorlinks=true,
    linkcolor=blue,
    filecolor=magenta,      
    urlcolor=blue,
}

\setlength{\parindent}{0pt}
\setlength{\parskip}{2pt}
\newcommand{\qspace}{\vskip15pt}
\newcommand{\hqspace}{\vskip5pt}



%%%%%%%%%%%%%%%%%%%%%%%%%%%%%%%%%%%%%%%%%%%%%%
\begin{document}


\begin{flushright}
UCSC Phys 133, Fall 2018\\
\end{flushright}

\begin{center}
\noindent  \textbf{ \large IC/HW 1: Physics Journal Article Reading Assignment}

Due: 12:01 AM, Tuesday 10/2. \\
Submit hard copy of responses to Schindler mailbox in ISB 232 Physics Mailroom.
\end{center}


Your primary task in this course will be to write reports based on lab experiments. Your reports will be written in the style of a 6-8 page scientific journal article. The goal of this reading assignment is to familiarize you with the tone and style of physics journals.

\hqspace

Browsing through the articles available to you, you will find that they have some common traits:

$\quad \bullet$ A well-thought-out structure with logical section layout.

$\quad \bullet$ Precise, unambiguous, clear statements. Every sentence is demonstrably true.

$\quad \bullet$ Well-chosen, useful figures. Figures have detailed explanatory captions.

$\quad \bullet$ Equations incorporated within sentences, as if to be read. Figures referenced in text.

$\quad \bullet$ Clear step-by-step reasoning and a clear overall message.

When writing your lab reports later in the quarter, if you get lost as to organization, tone, or style, check back in with some of these real journal articles for inspiration.

\hqspace

You've probably learned that all papers/lab reports have the same outline (Intro, Hypothesis, Methods, Results, etc.). Scanning through these papers, you'll quickly see that in physics, that's not true. One of the most important parts of writing a paper is deciding how to organize it into sections and subsections, and deciding what figures and data to show. Keep in mind while reading that there are no right answers, these are all decisions the authors need to make.

\qspace

1. Read the article \textit{Observation of Gravitational Waves from a Binary Black Hole Merger, LIGO Collaboration, 2016}, available here: \url{https://arxiv.org/abs/1602.03837}. \\
It's OK not to completely understand the article, but give it your best shot.

\qspace

2. Browse through and skim some other physics articles of your choosing. Good places to find articles include the the arXiv (\url{https://arxiv.org}), the APS journals sites (\url{https://journals.aps.org/about}), and our course article repository Google Drive folder (\href{https://drive.google.com/open?id=1Dozq9Of8CIdrI0FArCtk6bdCjGIkfbbB}{link}).


\qspace

3. After browsing a bit, choose one other article to read on a topic that interests you. (Must be at least four pages and contain at least two figures. If you are interested in a very long article, feel free to skim parts of it.) \\
It's OK not to completely understand the article, but give it your best shot.


\qspace

4. For each of the two articles you've read, give a brief report including:

$\quad \bullet$ The article's Authors, Title, Year, and (if available) Journal.

$\quad \bullet$ A one-paragraph summary of the article, in your own words.

$\quad \bullet$ An outline-style list of the sections and subsections in the article.

$\quad \;\;$ Give a very short explanation of the content and importance of each part.

$\quad \bullet$ A list of all figures/tables in the article. 

$\quad \;\;$ For two of them, give a very short explanation of the content and importance of each.





\end{document}