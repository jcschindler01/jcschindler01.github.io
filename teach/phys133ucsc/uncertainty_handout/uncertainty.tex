\documentclass[11pt]{article}

\usepackage[top=1.25in, bottom=1in, left=.8in, right=.8in]{geometry}

\usepackage{amsmath,amssymb,amsthm}
\usepackage{graphicx}
\usepackage{mwe}



%%%%%%%%%%%%%%%%%%%%%%%%%%%%%%%%%%%%%%%%%%%%%%
\begin{document}

\begin{flushright}
2018 SCHINDLER UCSC PHYS 133
\end{flushright}

\begin{center}
\noindent  \textbf{ \LARGE Uncertainty and Systematic Error in Measurement}
\end{center}

\textbf{Every measurement is imperfect.} Analysis of the ways in which a measurement is imperfect often goes by the title ``Error Analysis". However, the existence of experimental error is not necessarily an indication that an ``error", in the colloquial sense, has been committed. After all, EVERY measurement is imperfect. \textbf{The experimenter's job is to understand and quantify the various contributions to imperfection of the measurement, and reduce them to the extent possible.}

Contributions to the imperfection of a measurement can usually be broken down into two categories: uncertainty, and systematic error.














\end{document}
